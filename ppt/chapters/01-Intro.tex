\section{Inroduction}
    %\frame{\sectionpage}

%%
\begin{frame}[fragile]{Motivation}

有使用過PTT的人都不陌生,每當地震後第一件要做的事情不是報平安,而是到八卦版上發地震文。

在2020/5/3上午11:29,我在臺大社科院感受到地震,幾秒後我突然好奇,若在這時打開PTT的八卦版,會看到幾篇地震文。結果已然有十數篇,更發現第一篇早已推爆,且發文時間在11:28:18。這意味者早在我感受到地震以前,在別處已有人感受到地震而發文。

這不但受地震的傳播速度影響,更讓我發想了以下的問題:究竟還有哪些因素影響著人們「能夠搶到PTT的地震爆文」呢?

\end{frame}

%%
\begin{frame}[fragile]{問題發想}

根據{\color{blue}\href{http://scimonth.blogspot.com/2009/09/blog-post_1815.html}{科學月刊}}的文章,P波每秒走5-7 km,S波每秒走3-4 km。而5/3上午的地震震央在台東臺北方海域,爆文則由來自台中的網友奪得。上一次在八卦版有文章的地震則是4/12,震央及爆文發文者均在宜蘭。
因此可以提出以下的問題:

- PTT使用者與震央的遠近是否會顯著地影響奪得爆文的機率

與此同時,我們必須進一步問的可能就是,如果會,那麼:

- 如果其他條件不變,我所處的縣市距離震央每遠一公里,會比別人多花幾秒鐘發文?

\end{frame}

%%
\begin{frame}[fragile]{問題發想}

為了要回答這個問題,我們顯然需要發文者的位置,幸好PTT已經紀錄了IP位址,我們只要反查即可。如:{\color{blue}\href{https://www.ez2o.com/App/Net/IP}{反查IP網站}}

以及利用中央氣象局發佈的地震報告當中的震央資訊,即可比對發文者位置與震央位址的遠近。

此外,我們還好奇,是否有「縣市別」的因素影響著PTT使用者是否搶到地震爆文。例如:人口密度、各縣市的平均移動網路速度等。

\end{frame}


%%
\begin{frame}[fragile]{Approach}

我們以如下的方式獲取資料:

\begin{enumerate}
	\item 以Request及BeautifulSoup爬PTT八卦版關鍵字有「地震」的文章
	\item 以Selenium反查IP位址,紀錄經緯度資訊
	\item 以Pandas套件清資料
\end{enumerate}

\end{frame}

