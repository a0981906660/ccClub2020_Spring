

\section{Conclusion}



%
\begin{frame}[fragile]{Questions Asked}
    \begin{enumerate}
        \item PTT使用者與震央的遠近是否會顯著地影響奪得爆文的機率
            \begin{itemize}
                \item 會
            \end{itemize}
        \item 如果其他條件不變,我所處的縣市距離震央每遠一公里,我搶到爆文的機率會低多少?
            \begin{itemize}
                \item 大約0.13\%
            \end{itemize}
        \item 如果其他條件不變,我所處的縣市距離震央每遠一公里,會比別人多花幾秒鐘發文?
            \begin{itemize}
                \item 大約4秒鐘
            \end{itemize}
    \end{enumerate}

\end{frame}


%
\begin{frame}[fragile]{Possible Approaches}

這份專題研究仍有不足之處,或可朝以下方向改進:
    \begin{enumerate}
        \item 累積更長期間的資料,以避免少數縣市幾乎沒有observation的問題
        \item 加入每起地震在各縣市的震度資料作為控制變數
        \item 少部分由IP定位至「鄉鎮」層級的地理資訊受限於絕大部分資料只能定位至「縣市」層級,無法得出其細節資訊
    \end{enumerate}

\end{frame}


%
\begin{frame}[fragile]{視覺化}

針對此次專案,我們所整理的資料結合地圖來視覺化。

\end{frame}
